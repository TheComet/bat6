Mit der  zunehmenden Verbreitung von  erneuerbaren Energiequellen hat  sich in
den letzten Jahren die Solartechnologie zunehmender Popularit\"at erfreut, und
wird  mit  grosser Wahrscheinlichkeit  in  Zukunft  noch weiter  an  Bedeutung
gewinnen.

Doch wie  mit vielen anderen  Technologien w\"achst mit  steigender Verwendung
auch bei der  Solartechnik das Bestreben, die Technologie  besser zu verstehen
und  schlussendlich besser  zu kontrollieren,  um ihr  Potenzial bestm\"oglich
aussch\"opfen zu k\"onnen.

Im  Rahmen  des  Projekts  4  soll im  Fr\"uhlingssemester  2016  deshalb  ein
\"Uberwachungsg\"at  f\"ur   Photovoltaik-Zellen  entwickelt   werden. Um  die
korrekte Funktionsweise  dieses \"Uberwachungsger\"ates im  Labor verifizieren
zu k\"onnen,  soll in  diesem Projekt ein  Netzger\"at entwickelt  werden, das
eine Photovoltaik-Zelle unter verschiedenen Bedingungen simulieren kann.

Als Zusatzanforderung  soll nicht nur  eine einzelne, sondern auch  die Serie-
und  Parallelschaltung  mehrerer  PV-Zellen  simuliert  werden  k\"onnen. Dazu
werden mehrere Simulatoren entsprechend gekoppelt.

Das Ger\"at  ist grunds\"atzlich  ein Labornetzger\"at, mit  der Besonderheit,
dass  verschiedene   Strom-Spannungs-Kurven  vom  Benutzer   definiert  werden
k\"onnen. Diese   unterschiedlichen  Kurvenverl\"aufe   orientieren  sich   am
Verhalten  von   PV-Modulen  unter   verschiedenen  Umst\"anden   (z.B.  volle
Sonneneinstrahlung, teilweise abgeschattet/verdreckt, etc.).

Das   Ger\"at    ist   mikrocontroller-basiert    und   hat    ein   einfaches
Benutzer-Interface  sowie  die  M\"oglichkeit,   mit  einem  PC  via  serielle
Schnittstelle zu kommunizieren.


TODO: Schaltregler
