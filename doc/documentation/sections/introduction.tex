% Structured according to Gertiser's feedback:
% 1) Worum geht es?
% 2) Ziel und Anforderungen
% 3) Umsetzung -> Grobkonzept
% 4) Resultate
% 5) Aufbau der Arbeit

% TODO: source for expensive shit

Verifying the correct operation of equipment  used for testing solar arrays is
currently difficult in  a controlled laboratory environment,  as no affordable
solutions for this exist on the open market. To address this issue, the aim of
this project was the development of a device capable of emulating the behavior
of a  photovoltaic module. With this  device, test equipment  for photovoltaic
modules can be verified to function  as specified in a laboratory conveniently
and affordably.

The primary characteristics  of our device should be compactness  (usable in a
standard  laboratory environment),  the  capability to  emulate varying  solar
irradiation levels,  being able  to be  used in  series with  other emulators,
effiency with  regards to  power consumption  and losses,  and to  emulate the
characteristic  curve of  an actual  PV module. A  more detailed  list of  the
technical  requirements  can be  found  in  the specifications  (see  appendix
\ref{appendix:specs} on page \pageref{appendix:specs}f)

A microcontroller and a constant-current, constant-voltage step-down converter
constitute the core of our device.  The microcontroller performs IO operations
and implements  the regulation loop  used to control the  step-down converter,
whereas  the  step-down converter  generates  the  output voltage  and  output
current  corresponding  to  the  desired  operating  point  from  a  DC  power
supply. The  entire design  is  based  around a  custom  PCB,  enabling us  to
optimise impedances of  connections between critical components. Additionally,
we can get very close in behavior to a potentially mass-produced product since
trace lenghts, routing  and component placement are crucial  (see also section
\ref{sec:verification}, beginning on page \pageref{sec:verification}).

The device has a simple yet powerful user interface which can be controlled by
a push-twist button and a modern OLED display. The device can also be remotely
monitored and controlled from a PC via USB interface and custom software built
on the  Qt application framework  and is  compatible with all  major operating
systems~\cite{ref:qt}.  Our  device can emulate fairly  complex configurations
of cells thanks to an efficient  use of computation resources and its powerful
microprocessor.

Section  \ref{sec:concept}   (p.  \pageref{sec:concept}ff)  of   this  reports
deal  with   the  basic  concept  behind   our  solution. Component  selection
and   PCB  design   are  documented   in  sections   \ref{sec:components}  (p.
\pageref{sec:components}ff)   and   \ref{sec:pcb}  (p.   \pageref{sec:pcb}ff),
respectively,   while    section   \ref{sec:software}   is    concerned   with
software  (p. \pageref{sec:software}ff).   Section \ref{sec:verification}  (p.
\pageref{sec:verification}ff)  deals with  testing  the device  and our  error
analysis. Lastly, the conclusion can  be found in section \ref{sec:conclusion}
on page \pageref{sec:conclusion}.
