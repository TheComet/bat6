% Structured according to Gertiser's feedback:
% 1) Worum geht es?
% 2) Ziel und Anforderungen
% 3) Umsetzung -> Grobkonzept
% 4) Resultate
% 5) Aufbau der Arbeit

The objective  of our next project  (\emph{Project 4}) will be  to construct a
device  capable  of  monitoring  photovoltaic  modules  used  in  solar  power
arrays. In order to  be able to test that device  in a laboratory environment,
the  aim of  this project  is to  construct a  device capable  of emulating  a
photovoltaic module.

The primary characteristics  of our device should be compactness  (usable in a
standard  laboratory environment),  the  capability to  emulate varying  solar
irradiation levels,  being able  to be  used in  series with  other emulators,
effiency with  regards to  power consumption  and losses,  and to  emulate the
voltage-current  line  of  an  actual  PV  module. A  more  detailed  list  of
the  required specifications  can  be  found in  appendix  \code{XXX (link  to
Lastenheft)}. %TODO

At  the core  of  our device  lie a  microcontroller  and a  constant-current,
constant-voltage step-down converter.  The  microcontroller is responsible for
programming and  controlling the  device, whereas  the step-down  converter is
used  to  generate the  output  voltage  and  output  current of  the  desired
operating point  from a DC power  supply. The entire design is  based around a
custom  PCB. This enables  us to  get very  close in  behavior to  a porential
mass-produced  product since  trace lenghts,  routing and  component placement
are  crucial  (see  also  section \ref{sec:verification},  beginning  on  page
\pageref{sec:verification}).

Unfortunately,  at this  point,  the  device is  not  fully operational  since
the  step-down  converter  keeps  being   damaged. We  have  been  working  on
diagnosing  the  cause  of  this  issue, but  fixing  it  would  require  more
time  than   we  have   at  our  disposal. For   more  information   on  this,
see  sections  \ref{sec:verification}  on  pages  \pageref{sec:verification}ff
and   our  concluding   remarks  in   section  \ref{sec:conclusion}   on  page
\pageref{sec:conclusion}.

Sections  \ref{sec:task}   (p.  \pageref{sec:task}ff)   and  \ref{sec:concept}
(p.   \pageref{sec:concept}ff)   of   this   report   deal   with   the   task
itself  and  the  basic   concept  behind  our  solution. Component  selection
and   PCB  design   are  documented   in  sections   \ref{sec:components}  (p.
\pageref{sec:components}ff)   and   \ref{sec:pcb}   (p.   \pageref{sec:pcb}ff)
respectively, while section \ref{sec:software}  is concerned with the software
(p.  \pageref{sec:software}ff).   Lastly, section  \ref{sec:verification}  (p.
\pageref{sec:verification}ff) deals  with testing  the device  including error
analysis, and the  conclusion can be found in  section \ref{sec:conclusion} on
page \pageref{sec:conclusion}.
