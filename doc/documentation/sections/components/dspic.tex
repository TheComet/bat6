The choice for our miroprocessor was comparatively straight-forward. Microchip
was  chosen as  manufacturer  because  one of  our  team  members was  already
familiar with their  products. This enabled us to more  quickly start becoming
productive instead of  the entire team needing to familiarise  itself with the
tools and architecture of the microcontroller from scratch.

Additionally, Microchip provides good developer tools for free. For selecting
a specific model, the following criteria eventually lead us to the \code{dsPIC33EP16GS506}:

\begin{itemize}
    \item
        \SI{120}{\mega\hertz} clock (60 MIPS): Comparatively fast. Allows for a
        fast control loop.
    \item
        2 ADCs  and 2  DACs with external  voltage reference: Required  by the
        control model we use (see section \code{TODO}.
    \item
        Programmable Gain Amplifier (64 $\times$ analog pre-amplifier): Allows
        measuring the small differential voltages which occur in our device.
    \item
        Low cost of 4 CHF
\end{itemize}
