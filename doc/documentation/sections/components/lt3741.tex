Die  Ausgangsspannung   muss   mindestens   im  Bereich  von  \SI{0}{\volt}  bis
\SI{24}{\volt}  liegen  und   einen  Rippel  kleiner  als  \SI{300}{\milli\volt}
besitzen.  Der  Ausgangsstrom muss mindestens im Bereich von \SI{0}{\ampere} bis
\SI{3.5}{\ampere} liegen  und  einen  Rippel kleiner als \SI{100}{\milli\ampere}
besitzen. Dabei  sollte  die Effizienz bei Volllast mindestens \SI{80}{\percent}
betragen.

Da  das  Endprodukt  schlussendlich  in  Serie  mit  mehreren   Spannungs-  oder
Stromquellen  geschalten  werden  k\"onnte, muss  zus\"atzlich  darauf  geachtet
werden,  dass  der Spannungswandler \emph{leistungsaufnahmef\"ahig}  sein  muss.
Diese Eigenschaft weist  ein  sogenannter  \emph{Synchronwandler}  vor und wurde
zusammen mit den Spannungs-,  Strom-,  und Leistungsanforderungen als prim\"ares
Merkmal f\"ur die Produktsuche eines Wandlers verwendet.

Der LT3741 ist einer der Bauteile,  die  alle  Anforderungen  erf\"ullt.  In der
Abbildung \ref{fig:circuit:buck} ist der Aufbau zu sehen.

\begin{figure}[th!]
    \center
    \includegraphics[width=.75\textwidth]{images/circuit/buck.pdf}
    \caption{Herzst\"uck des Projektes: Aufbau des LT3741 CVCC Synchronwandler}
    \label{fig:circuit:buck}
\end{figure}

% **************************************************************************** %
\subsubsection{St\"utzkondensatoren}
% **************************************************************************** %

\input{sections/components/lt3741/kondensitator}

% **************************************************************************** %
\subsubsection{Schaltfrequenz}
% **************************************************************************** %

\input{sections/components/lt3741/schaltfrequenz}

% **************************************************************************** %
\subsubsection{Auswahl Spule}
% **************************************************************************** %

Die Spule $L_1$, ersichtlich in der Abbildung \ref{fig:circuit:buck}, wurde  mit
der Formel \ref{eq:circuit:buck:inductor} berechnet,

\begin{equation}
    L_1 = \left( \frac{U_{in} \cdot U_{out} - U_{out}^2}{0.3 \cdot f_S \cdot I_O \cdot U_{in}} \right) = \SI{6}{\micro\henry}
    \label{eq:circuit:buck:inductor}
\end{equation}

wobei  $U_{in}$  die  Eingangsspannung von  \SI{36}{\volt}  ist,  $U_{out}$  die
Ausgangsspannung  bei  gr\"osster   Leistung  ist  (\SI{18}{\volt}),  $f_S$  die
Schaltfrequenz von  \SI{1}{\mega\hertz} ist und $I_o$ der maximale Ausgangsstrom
von \SI{5}{\ampere} ist.

Um den Rippel  noch ein wenig mehr zu gl\"atten, wurde eine gr\"ossere Spule von
\SI{22}{\micro\henry} ausgew\"ahlt.

Der Maximalstrom durch die Spule ist gleich gross wie  der  Strom, der durch den
MOSFET   fliesst   und   wird  mit  der  Formel  \ref{eq:circuit:buck:mosfet_id}
berechnet.  Der  S\"attigungsstrom   wurde   mit   $1.2   \cdot   I_{D_{max}}  =
\SI{6.2}{\ampere}$ berechnet. Es  wurde  nach passende Spulen gesucht, welche in
der Tabelle \ref{tab:circuit:buck:inductor} eingetragen sind.

\begin{table}[th!]
    \begin{center}
        \caption{}
        \label{tab:circuit:buck:inductor}
        \begin{tabular}{lcccc}
            \toprule
            Digikey         & Price (CHF) & Inductance (\SI{}{\micro\henry}) & DCR (\SI{}{\ohm}) & Ohmic Loss (\SI{}{\watt}) \\
            \midrule
            \rowcolor{lightgray}
            732-4237-1-ND   & 8.03        & 22                               & 0.007             & 0.175  \\
            732-2179-1-ND   & 6.4         & 47                               & 0.0335            & 0.8375 \\
            732-2177-1-ND   & 6.4         & 22                               & 0.0146            & 0.365  \\
            \bottomrule
        \end{tabular}
    \end{center}
\end{table}

Unter den Kandidaten ist ganz klar wegen des niedrigen DCRs die Erste,  mit Grau
hervorgehobene Spule, die optimalste.


% **************************************************************************** %
\subsubsection{Auswahl MOSFETs}
% **************************************************************************** %

In contrast to a non-synchronous regulator, this design uses  two  complementary
MOSFETs $V_2$ and $V_3$, where $V_3$ acts as an active replacement for the  free
wheeling diode typically found in non-synchronous designs. As mentioned earlier,
a crucial feature of  this  device  is the ability to \emph{absorb} power. $V_3$
makes  this  possible  because  it  is  able to regulate current in the opposite
direction through the inductor $L_1$.

When  selecting  switching  MOSFETs,  the  following  parameters are critical in
determining the best devices for a given application: $Q_G$ (Total Gate Charge),
$R_{DS_{(on)}}$ (On-Resistance), $Q_{GD}$ (Gate to Drain Charge), $Q_{GS}$ (Gate
to  Source  Charge),  $R_G$  (Gate  Resistance),  sowie  $U_{GS}$  und $U_{DS}$,
$I_{D_{max}}$ and $U_{GS_{THR}}$.

The maximum drain current is equal to the previously  calculated  peak  inductor
current $I_{L_{1_{peak}}}$ in equation \ref{eq:circuit:buck:inductor_peak}.
\begin{equation}
    I_{D_{max}} = I_{L_{1_{peak}}} = I_O + \left(\frac{U_{in}\cdot U_{out} - U_{out}^2}{2\cdot f_S \cdot L_1 \cdot U_{in}}\right) = \SI{5.2}{\ampere}
    \label{eq:circuit:buck:mosfet_id}
\end{equation}
where $U_{in}$ is the  input  voltage  \SI{28}{\volt},  $U_{out}$  is the output
voltage at peak power (which exists at $U_{out} = \SI{14}{\volt}$), $f_S$ is the
switching  frequency  \SI{1}{\mega\hertz},  $L_1$  is the value of the  selected
inductor (\SI{22}{\micro\henry})  and  $I_O$  is  the  maximum  output  current,
assumed to be $I_O = \SI{5}{\ampere}$.

The  maximum drain-to-source voltage $U_{DS}$ must be  greater  than  the  input
voltage $U_{in} = \SI{28}{\volt}$,  including  transients.  We  selected MOSFETs
with $U_{DS} = \SI{40}{\volt}$.

The  signals  driving  the  gates  of  the MOSFETs have  a  maximum  voltage  of
\SI{5}{\volt}  with   respect  to  the  source.  During  start-up  and  recovery
conditions, the gate drive signals  may  be  as  low as \SI{3}{\volt}. To ensure
that the LT3741 recovers properly, the  maximum gate threshold voltage should be
less than \SI{2}{\volt}. For a robust design, the maximum gate-to-source voltage
$U_{GS}$ should be greater than \SI{7}{\volt}.

Power losses in the MOSFETs are related to the on-resistance $R_{DS{(on)}}$; the
transition  losses  related  to  the  gate   resistance   $R_G$;   gate-to-drain
capacitance  $Q_{GD}$ and gate-to-source capacitance $Q_{GS}$. Power loss to the
on-resistance is an  Ohmic loss, $I^2 R_{DS_{(on)}}$. The power loss in the high
side     MOSFET     $V_2$      can      be     approximated     with     formula
\ref{eq:circuit:buck:mosfet_ploss}.

\begin{multline}
    P_{LOSS} = (\textrm{ohmic loss}) + (\textrm{transission loss}) \\
             \approx \left( I_O^2 \cdot R_{DS_{(on)}} \cdot \rho_T \right)
                    + \left( \frac{U_{in} \cdot I_O}{\SI{5}{\volt}} \cdot \left(Q_{GD} + Q_{GS} \right) \cdot \left( 2 \cdot R_G + R_{PU} + R_{PD} \right) \cdot f_S \right) \\
    \label{eq:circuit:buck:mosfet_ploss}
\end{multline}
where  $\rho_T$ is a temperature-dependant term of the  MOSFET's  on-resistance.
Using \SI{70}{\degree C} as the maximum operating temperature, $\rho_T$ is
roughly  equal  to  $1.3$.  $R_{PD}$  and $R_{PU}$ are the LT3741 high side gate
driver   output   empedance,  \SI{1.3}{\ohm}  and  \SI{2.3}{\ohm}  respectively.

Another power loss related to switching MOSFET selection  is  the  power lost to
driving the gates. The total gate  charge, $Q_G$, must be charged and discharged
each switching cycle. The power is lost to the  internal  LDO within the LT3741.
the power lost to the charging of the gates is:
\begin{equation}
    P_{LOSS\_LDO} \approx \left( (U_{in} - \SI{5}{\volt} \right) \cdot \left( Q_{GLG} + Q_{GHG} \right) \cdot f_S
    \label{eq:circuit:buck:switching_loss}
\end{equation}
where $G_{GLG}$ is the low side gate charge  and $Q_{GHG}$ is the high side gate
charge.

In the table \ref{tab:circuit:buck:mosfet} are various candidates  that meet the
above  constraints.  For   each   candidate  the  power  losses  $P_{LOSS}$  and
$P_{LOSS\_LDO}$ was calculated.

\begin{table}[th!]
    \begin{center}
        \caption{}
        \label{tab:circuit:buck:mosfet}
        \begin{tabular}{cccccccccc}
            \toprule
            $R_{DS_{(on)}}$ & $Q_{GD}$ & $Q_{GS}$ & $R_G$ & $U_{GS_{THR}}$ & Ohmic Loss & Transision Loss & Total Loss & Drive Loss \\
            \midrule
            0.0032          & 4        & 2.5      & 0.4   & 2.5            & 0.104      & 1.0296          & 1.1336     & 0.806 \\
            0.0039          & 7        & 9        & 2.4   & 3.3            & 0.12675    & 4.8384          & 4.96515    & 1.984 \\
            0.0042          & 7        & 9        & 2.4   & 3.3            & 0.1365     & 4.8384          & 4.9749     & 1.984 \\
            0.008           & 2        & 4.5      & 3     & 2              & 0.26       & 2.2464          & 2.5064     & 0.558 \\
            0.0067          & 5.3      & 3.9      & 1.5   & 1              & 0.21775    & 2.18592         & 2.40367    & 0.7998 \\
            \rowcolor{lightgray}
            0.0093          & 2        & 4.9      & 1     & 2              & 0.30225    & 1.39104         & 1.69329    & 1.488 \\
            0.019           & 8        & 4        & 1.3   & 2              & 0.6175     & 2.6784          & 3.2959     & 1.798 \\
            0.0095          & 7.5      & 6        & 1     & 3              & 0.30875    & 2.7216          & 3.03035    & 1.736 \\
            \bottomrule
        \end{tabular}
    \end{center}
\end{table}

The MOSFET highlighted in grey is the one we selected. In  this case, it is only
the second best candidate that fits the required parameters, but  it  is  a  lot
cheaper than the best fit and has better documentation.

The same device is used for both the high-side and low-side switch.



% **************************************************************************** %
\subsubsection{Spannungs- und Strommessung}
% **************************************************************************** %



Der  LT3741  ist  sowohl   Spannungsgesteuert   wie   auch  Stromgesteuert.  Der
Spannungsteiler  $R_{11} \parallel R_2$ (siehe Abbildung  \ref{fig:circuit:buck}
oder    Abbildung    \ref{fig:circuit:buck:uset})   erlaubt   das   Messen   der
Ausgangsspannung  und  ein  Shunt-Widerstand  $R_4$   erm\"oglicht   die  genaue
\"Uberwachung des Stromes durch die Spule  $L_1$.  Der Widerstand $R_4$ wurde so
gew\"ahlt damit der  maximale  Ausgangsstrom  maximal  \SI{5}{\ampere}  betragen
kann.

Strom\"uberwachung   ist  sehr  wichtig  bei  einer  solchen  Aufgabe   wo   die
Ausgangsspannung  sich  konstant  \"andert.  Sie erlaubt  genauer  vorhersebares
Verhalten der Spannungs\"anderung am Ausgang -- \"Uberschiessen der Sollspannung
und  extreme  Stromspitzen  in  der  Spule  k\"onnen  besser  vermieden  werden.

Weiter   kann   ein   Stromgesteuerter   Regler  auch  als   Konstantstromquelle
funktionieren. Diese Eigenschaft  ist  vorallem  dann  von  Bedeutung  wenn  der
Arbeitspunkt  sich  im  ``steilen''  bereich  der   UI-Kennlinie  des  PV-Moduls
befindet.

Die  Feedback-Widerst\"ande   $R_2$   und   $R_{11}$   wurden  nach  der  Formel
\ref{eq:circuit:buck:feedback_resistors}       dimensioniert      damit      die
Ausgangsspannung maximal \SI{23}{\volt} betr\"agt.

\begin{equation}
    U_{out} = \SI{1.21}{\volt} \left( 1 + \frac{R_{11}}{R_2} \right)
    \label{eq:circuit:buck:feedback_resistors}
\end{equation}

Die Ausgangsspannung kann  danach  durch Anheben der Bezugsspannung $BUCK\_USET$
nach der  Formel \ref{eq:circuit:buck:uset} ver\"andert werden. 

\begin{equation}
    U_{out} = (\SI{1.21}{\volt} - BUCK\_USET) \cdot \frac{R_{11} + R_2}{R_2}
    \label{eq:circuit:buck:uset}
\end{equation}

Wobei $BUCK\_USET$ die analoge Spannung vom  ersten  DAC  ist.  In der Abbildung
\ref{fig:circuit:buck:uset} ist die dazugeh\"orige Schaltung.

\begin{figure}[th!]
    \center
    \includegraphics[width=.35\textwidth]{images/circuit/buck-uset.pdf}
    \caption{Einstellung der Ausgangsspannung durch \"Anderung der Bezugsspannung im Feedback-Loop mittels einer analogen Steuerspannung von 0V bis 1.21V}
    \label{fig:circuit:buck:uset}
\end{figure}

Analog zur Ausgangsspannung kann auch der Maximalstrom eingestellt werden. Durch
anlegen einer  analogen  Spannung  zwischen \SI{0}{\volt} und \SI{1.5}{\volt} am
Eingang  CTRL1 des LT3741 kann  direkt  der  maximale  \emph{Durchschnittsstrom}
durch die Spule $L_1$ und  somit  der maximale Ausgangsstrom eingestellt werden.

\begin{figure}[th!]
    \center
    \includegraphics[width=.4\textwidth]{images/circuit/buck-iset.pdf}
    \caption{Einstellung des Maximalstroms mittels einer analogen Steuerspannung von 0V bis 1.5V}
    \label{fig:circuit:buck:iset}
\end{figure}

Die Abbildung \ref{fig:circuit:buck:iset} zeigt  die  dazugeh\"orige  Schaltung.
Der   maximale  durchschnittliche  Ausgangsstrom  $I_o$  wird  mit  der   Formel
\ref{eq:circuit:buck:output_current} berechnet

\begin{equation}
    I_o = \frac{U_{CTRL1}}{30 \cdot R_4}
    \label{eq:circuit:buck:output_current}
\end{equation}

wobei $U_{CTRL1}$ die  analoge  Steuerspannung vom zweiten DAC ist und $R_4$ der
\SI{10}{\milli\ohm}    Shunt-Widerstand    ist,   welcher   in   der   Abbildung
\ref{fig:circuit:buck} zu sehen ist.

Damit der Mikrocontroller angemessene Steuerspannungen  generieren kann, braucht
er die Ausgangsspannung und den Ausgangsstrom zu messen.

Die   Ausgangsspannung   wird   mittels   der   Schaltung   in   der   Abbildung
\ref{fig:circuit:buck:umeas} gemessen. Die Widerst\"ande $R_{12}$  und  $R_{15}$
wurden  so  dimensioniert  damit  die  Spannung  $BUCK\_UMEAS$  im  Bereich  von
\SI{0}{\volt} bis \SI{1.5}{\volt} skaliert ist.

\begin{figure}[th!]
    \center
    \includegraphics[width=.45\textwidth]{images/circuit/buck-umeas.pdf}
    \caption{Messen der Ausgangsspannung}
    \label{fig:circuit:buck:umeas}
\end{figure}

Der  Ausgangsstrom  wird  mittels  einem  Shunt-Widerstand  $R_5$  differentiell
gemessen.  Die  Schaltung dazu ist in der Abbildung \ref{fig:circuit:buck:imeas}
dargestellt.

\begin{figure}[th!]
    \center
    \includegraphics[width=.85\textwidth]{images/circuit/buck-imeas.pdf}
    \caption{Messen des Ausgangsstromes}
    \label{fig:circuit:buck:imeas}
\end{figure}

Es  ist  zu  beachten,  dass  die  Widerst\"ande  $R_{10}$  und  $R_{14}$  einen
Bias-Strom durch den Widerstand $R_5$  verursachen.  Somit  entsteht ein kleiner
Spannungs-Offset.
\begin{equation}
    U_{offset} = \frac{ \SI{3.3}{\volt} \cdot R_5 }{ R_{14} + R_{10} + R_5 }
    \label{eq:circuit:buck:shunt_offset}
\end{equation}

Da   der   ADC   eine   12-bit   Aufl\"osung   mit  einer  Referenzspannung  von
\SI{3.3}{\volt} hat, gilt:
\begin{equation}
    U_{step} = \frac{\SI{3.3}{\volt}}{2^{12}} = \SI{806}{\micro\volt}
    \label{eq:circuit:buck:adc_step}
\end{equation}

Die Widerst\"ande $R_9$, $R_{10}$, $R_{10}$ und  $R_{14}$  sollten  so klein wie
m\"oglich  dimensioniert  werden damit St\"orungen an  den  Leitungen  minimiert
werden  k\"onnen,  aber  sollten immer noch gross genug sein, damit  $U_{offset}
\leq  U_{step}$.  Zu  gross  d\"urfen  sie  auch  nicht  sein,  weil  sonst  die
Holding-Time  des   ADCs   nicht   mehr   erf\"ullt   ist  (was  bei  ca.  $\geq
\SI{5}{\kilo\ohm}$      der      Fall     ist).     Aus     den      Gleichungen
\ref{eq:circuit:buck:shunt_offset} und \ref{eq:circuit:buck:adc_step} kann jetzt
nach den 4 Widerst\"anden aufgel\"ost werden. Es gilt:
\begin{align*}
                          U_{step} &\geq U_{offset} \\
    \frac{\SI{3.3}{\volt}}{2^{12}} &\geq \SI{3.3}{\volt} \cdot \frac{R_5}{R_x + R_5} \\
                  \frac{1}{2^{12}} &\geq \frac{R_5}{R_x + R_5} \\
                               R_x &\geq \left( 2^{12} - 1 \right) \cdot R_5 \\
\end{align*}

wobei  $\frac{R_x}{2}  =  R_{9}  =  R_{10} = R_{13} = R_{14}$. Berechnet  ergibt
$\frac{R_x}{2} \approx \SI{22}{\ohm}$.

Eine weitere Einschr\"ankung, vorallem bei  kleinen  Widerst\"anden,  ist,  dass
nicht  unn\"otig  viel  Leistung  verbraten  werden sollte. Deshalb  werden  die
Widerst\"ande ein wenig h\"oher mit \SI{270}{\ohm} dimensioniert. In diesem Fall
ist der Leistungsverlust aller 4 Widerst\"ande:
\begin{equation*}
    P_{loss} \approx \frac{\SI{3.3}{\volt}^2}{2\cdot \SI{270}{\ohm}} \approx \SI{20}{\milli\watt}
\end{equation*}

Die gemessene Spannung  am  Shunt-Widerstand  ist recht klein. Deshalb verwenden
wir  den  im  Mikrocontroller  eingebauten  vorverst\"arker  (PGA),   was   eine
Verst\"arkung von bis  zu Faktor 64 erreichen kann. Das verst\"arkte Signal wird
intern an der eingebauten differentiellen ADC weitergeleitet.



% **************************************************************************** %
\subsubsection{Ausgang}
% **************************************************************************** %

Die Ausgangsspannung wird \"uber zwei  Bananenstecker  $X_{6A}$ und $X_{6B}$ ans
\"Aussere    des    Geh\"auses     gef\"uhrt.     Die    Ausgangsspannung    ist
verpolungsgesch\"utzt mit der Diode $V_4$.

\begin{figure}[th!]
    \center
    \includegraphics[width=.35\textwidth]{images/circuit/output-connectors.pdf}
    \caption{Verpolungsschutz am Ausgang}
    \label{fig:circuit:output}
\end{figure}

Damit die ADCs und DACs  m\"oglichst  genau messen und m\"oglichst in Full-Range
betrieben    werden   k\"onnen,   wird   eine   externe   Referenzspannung   von
\SI{1.5}{\volt}    verwendet     (siehe    Abbildung    \ref{fig:circuit:vref}).

\begin{figure}[th!]
    \center
    \includegraphics[width=.4\textwidth]{images/circuit/vref.pdf}
    \caption{1.5V Referenzspannung um die ADCs m\"oglichst in Full-Range betreiben zu k\"onnen}
    \label{fig:circuit:vref}
\end{figure}


% **************************************************************************** %
\subsubsection{Enable und UVLO}
% **************************************************************************** %

The   LT3741's  \emph{Enable}   input   is  enabled   and   disabled  by   the
microcontroller's  $BUCK\_EN$  signal on  one  hand,  on  the other  hand,  it
can  be disabled  in  hardware in  case the  \SI{28}{\volt}  rail drops  below
\SI{25}{\volt}. This  allows for  controlled and  predictable behavior  of the
LG3741 during power-on and  power-off processes. The corresponding circuit can
be found in figure \ref{fig:circuit:uvlo}.

\begin{figure}[th!]
    \center
    \includegraphics[width=.6\textwidth]{images/circuit/uvlo.pdf}
    \caption{Under-Voltage Lock-Out (UVLO) allows for controlled power-on and power-off of the controller}
    \label{fig:circuit:uvlo}
\end{figure}

In  case   of  under-voltage,  $N_6$   switches  to  locking  mode,   and  the
transistor $V_6$  starts to conduct,  thus pulling the \emph{Enable}  input to
\emph{Low}. Voltage $BUCK\_UVLO$ triggers an interrupt in the microcontroller.


