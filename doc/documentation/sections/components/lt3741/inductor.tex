The size of the inductor $L_1$, as illustrated in figure \ref{fig:circuit:buck},
was calculated using formula \ref{eq:circuit:buck:inductor}
\begin{equation}
    L_1 = \left( \frac{U_{in} \cdot U_{out} - U_{out}^2}{0.3 \cdot f_S \cdot I_O \cdot U_{in}} \right) = \SI{6}{\micro\henry}
    \label{eq:circuit:buck:inductor}
\end{equation}
where $U_{in}$ is the  input  voltage  \SI{28}{\volt},  $U_{out}$  is the output
voltage at peak power (which exists at $U_{out} = \SI{14}{\volt}$), $f_S$ is the
switching frequency \SI{1}{\mega\hertz} and $I_o$ is the maximum output current,
assumed   to   be   $I_o  =  \SI{5}{\ampere}$  for   some   additional   leeway.

We ended up selecting a  larger  inductor  of  $L_1  = \SI{22}{\micro\henry}$ to
further decrease ripple current.

In addition to the value of  the  inductor, the maximum current rating, DCR, and
saturation  current are also important factors to consider. The maximum  current
of the inductor is calculated using formula 
\ref{eq:circuit:buck:inductor_peak}
\begin{equation}
    I_{L_{1_{peak}}} = I_O + \left( \frac{U_{in} \cdot U_{out} - U_{out}^2}{2 \cdot f_S \cdot L_1 \cdot U_{in}} \right) = \SI{5.2}{\ampere}
    \label{eq:circuit:buck:inductor_peak}
\end{equation}
Where $L$  is  the  value  of  the selected inductor, \SI{22}{\micro\henry}. The
saturation  current  of the inductor was sized factor $1.2$ higher than the peak
current.
\begin{equation}
    I_{L_{1_{saturation}}} = 1.2 \cdot I_{L_{1_{peak}}}
    \label{eq:circuit:buck:inductor_saturation}
\end{equation}

A  list  of  candidates  matching  the  above  parameters  are  listed in  table
\ref{tab:circuit:buck:inductor}.

\begin{table}[th!]
    \begin{center}
        \caption{}
        \label{tab:circuit:buck:inductor}
        \begin{tabular}{lcccc}
            \toprule
            Digikey         & Price (CHF) & Inductance (\SI{}{\micro\henry}) & DCR (\SI{}{\ohm}) & Ohmic Loss (\SI{}{\watt}) \\
            \midrule
            \rowcolor{lightgray}
            732-4237-1-ND   & 8.03        & 22                               & 0.007             & 0.175  \\
            732-2179-1-ND   & 6.4         & 47                               & 0.0335            & 0.8375 \\
            732-2177-1-ND   & 6.4         & 22                               & 0.0146            & 0.365  \\
            \bottomrule
        \end{tabular}
    \end{center}
\end{table}

It  is  clear that the one with the lowest DCR will be the most optimal. The one
highlighted in grey is the one we chose.

