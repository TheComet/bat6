The size of the inductor $L_1$, as illustrated in Figure \ref{fig:circuit:buck},
was calculated using the formula below:

\begin{equation}
    L_1 = \left( \frac{V_{in} \cdot V_{out} - V_{out}^2}{0.3 \cdot f_S \cdot I_O \cdot V_{in}} \right) = \SI{6}{\micro\henry}
    \label{eq:circuit:buck:inductor}
\end{equation}

where  $V_{in}$ equals  the  input voltage  \SI{28}{\volt},  $V_{out}$ is  the
output voltage  at peak  power (which exists  at $V_{out}  = \SI{14}{\volt}$),
$f_S$ is the switching frequency  \SI{1}{\mega\hertz} and $I_O$ is the maximum
output  current, assumed  to be  $I_O  = \SI{5}{\ampere}$,  allowing for  some
additional leeway.   A larger  inductor of  $L_1 =  \SI{22}{\micro\henry}$ was
selected to further decrease ripple current.

In addition to the value inductor's value, the maximum current rating, DCR, and
saturation  current are also important factors to consider. The inductor's peak
current is calculated using
\begin{equation}
    I_{L_{1_{peak}}} = I_O + \left( \frac{V_{in} \cdot V_{out} - V_{out}^2}{2 \cdot f_S \cdot L_1 \cdot V_{in}} \right) = \SI{5.2}{\ampere}
    \label{eq:circuit:buck:inductor_peak}
\end{equation}

Where  $V_{in}$ equals  the  input voltage  \SI{28}{\volt},  $V_{out}$ is  the
output voltage  at peak  power (which exists  at $V_{out}  = \SI{14}{\volt}$),
$f_S$ is  the switching frequency  \SI{1}{\mega\hertz}, $L_1$ is the  value of
the selected inductor (\SI{22}{\micro\henry}) and  $I_O$ is the maximum output
current, assumed  to be  $I_O =  \SI{5}{\ampere}$.  The  inductor's saturation
current  was  sized  $1.2$  times  higher than  the  peak  current. With  this
defined,  a list  of  possible inductors  could be  compiled,  shown in  table
\ref{tab:circuit:buck:inductor}.

%\begin{equation}
%    I_{L_{1_{saturation}}} = 1.2 \cdot I_{L_{1_{peak}}}
%    \label{eq:circuit:buck:inductor_saturation}
%\end{equation}


\begin{table}[th!]
    \begin{center}
        \caption{List of inductors matching our requirements}
        \label{tab:circuit:buck:inductor}
        \begin{tabular}{lcccc}
            \toprule
            Digikey         & Price (CHF) & Inductance (\SI{}{\micro\henry}) & DCR (\SI{}{\ohm}) & Ohmic Loss (\SI{}{\watt}) \\
            \midrule
            \rowcolor{lightgray}
            732-4237-1-ND   & 8.03        & 22                               & 0.007             & 0.175  \\
            732-2179-1-ND   & 6.4         & 47                               & 0.0335            & 0.8375 \\
            732-2177-1-ND   & 6.4         & 22                               & 0.0146            & 0.365  \\
            \bottomrule
        \end{tabular}
    \end{center}
\end{table}

We  chose the  inductor highlighted  in  grey because  it has  the lowest  DCR
(direct current resistance) of the three, making it the most optimal choice.
