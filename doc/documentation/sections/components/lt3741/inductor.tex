The   size    of   the    inductor   $L_1$,    as   illustrated    in   Figure
\ref{fig:circuit:buck}  on  the fold-out,  was  calculated  using the  formula
below:

\begin{equation}
    L_1 = \left( \frac{V_{in} \cdot V_{out} - V_{out}^2}{0.3 \cdot f_S \cdot I_O \cdot V_{in}} \right) = \SI{6}{\micro\henry}
    \label{eq:circuit:buck:inductor}
\end{equation}

where  $V_{in}$ equals  the  input voltage  \SI{28}{\volt},  $V_{out}$ is  the
output voltage  at peak  power (which exists  at $V_{out}  = \SI{14}{\volt}$),
$f_S$ is the switching frequency  \SI{1}{\mega\hertz} and $I_O$ is the maximum
output  current, assumed  to be  $I_O  = \SI{5}{\ampere}$,  allowing for  some
additional leeway.   A larger  inductor of  $L_1 =  \SI{22}{\micro\henry}$ was
selected to further decrease ripple current.

In addition to the inductance, the maximum current rating, DCR, and saturation
current are also important factors to consider. The inductor's peak current is
calculated using
\begin{equation}
    I_{L_{1_{peak}}} = I_O + \left( \frac{V_{in} \cdot V_{out} - V_{out}^2}{2 \cdot f_S \cdot L_1 \cdot V_{in}} \right) = \SI{5.2}{\ampere}
    \label{eq:circuit:buck:inductor_peak}
\end{equation}

Where  $V_{in}$ equals  the  input voltage  \SI{28}{\volt},  $V_{out}$ is  the
output voltage  at peak  power (which exists  at $V_{out}  = \SI{14}{\volt}$),
$f_S$ is  the switching frequency  \SI{1}{\mega\hertz}, $L_1$ is the  value of
the selected inductor (\SI{22}{\micro\henry}) and  $I_O$ is the maximum output
current, assumed  to be  $I_O =  \SI{5}{\ampere}$.  The  inductor's saturation
current  was  sized  $1.2$  times  higher than  the  peak  current. With  this
defined,  a list  of  possible inductors  could be  compiled,  shown in  Table
\ref{tab:circuit:buck:inductor} in  appendix \ref{appendix:inductors}  on page
\pageref{appendix:inductors}.   We chose  the \emph{732-4237-1-ND}  because it
has the lowest DCR (direct current resistance) of the models listed.
