Die  Bauteile  $V_1$  und  $C_6$  bilden  zusammen  mit  dem  MOSFET  $V_2$  ein
High-Side-Schalter. Im Gegensatz zu einem nicht-synchronen Schaltregler befindet
sich  an  der Stelle wo normalerweise eine Freilaufdiode sein sollte ein zweiter
MOSFET $V_3$. Dieser erm\"oglicht  eine  Spannungsregelung in der gegengesetzten
Richtung -- sprich, sie erm\"oglicht eine Leistungsaufnahme, was, wie oben schon
erw\"ant wurde, kritisch ist.

Bei  der  Auswahl von passenden MOSFETs sind die  Parameter  $Q_G$  (Total  Gate
Charge),  $R_{DS_{(on)}}$  (On-Resistance),  $Q_{GD}$  (Gate to  Drain  Charge),
$Q_{GS}$ (Gate to Source  Charge),  $R_G$  (Gate Resistance), sowie $U_{GS}$ und
$U_{DS}$, $I_{D_{max}}$ und $U_{GS_{THR}}$ kritische Parameter.

Der  maximale  Drain-Strom  kann mit der Formel  \ref{eq:circuit:buck:mosfet_id}
berechnet werden

\begin{equation}
    I_{D_{max}} = I_O + \left( \frac{U_{in} \cdot U_{out} - U_{out}^2}{2 \cdot f_S \cdot L \cdot U_{in}} \right) = \SI{5.2}{\ampere}
    \label{eq:circuit:buck:mosfet_id}
\end{equation}

wobei $I_O$ der maximale Ausgangsstrom  von  \SI{5}{\ampere}  ist,  $U_{in}$ die
Eingangsspannung von  \SI{36}{\volt}  ist,  $U_{out}$  die  Ausgangsspannung bei
gr\"osster  Leistung  ist   (\SI{18}{\volt}),   $f_S$   die  Schaltfrequenz  von
\SI{1}{\mega\hertz}  ist  und  $L$  die  Spule  von  \SI{22}{\micro\henry}  ist.

$U_{DS}$   wurde   h\"oher  gew\"ahlt  als  die  Eingangsspannung:   $U_{DS}   >
\SI{36}{\volt}$.

Der LT3741 liefert als maximale Gate-Steuer-Spannung $U_{GS}$  \SI{5}{\volt}. Da
der LT3741 w\"ahrend dem Aufstartvorgang Steuersignale knapp unter \SI{3}{\volt}
liefert,   muss   die   Gate-Threshold-Spannung   $U_{GS_{THR}}$   kleiner   als
\SI{2}{\volt} gew\"ahlt werden. $U_{GS_{min}}$ muss gr\"osser  als \SI{5}{\volt}
sein.

Leistungsverluste der MOSFETs sind einerseits verbunden mit ohmsche  Verluste --
abh\"angig  von  $R_{DS_{(on)}}$  --  sowie   verbunden  mit  Schaltverluste  --
abh\"angig von $Q_{GS}$ und $Q_{GD}$.

Der    Leistungsverlust    im    High-Side   MOSFET   kann   mit   der    Formel
\ref{eq:circuit:buck:mosfet_ploss} approximiert werden

\begin{multline}
    P_{LOSS} = (\textrm{ohmic loss}) + (\textrm{transission loss}) \\
             \approx \left( I_O^2 \cdot R_{DS_{(on)}} \cdot \rho_T \right)
                    + \left( \frac{U_{in} \cdot I_O}{\SI{5}{\volt}} \cdot \left(Q_{GD} + Q_{GS} \right) \cdot \left( 2 \cdot R_G + R_{PU} + R_{PD} \right) \cdot f_S \right) \\
    \label{eq:circuit:buck:mosfet_ploss}
\end{multline}

wobei $\rho_T$ ein temperaturabh\"angiger Parameter vom Einschaltwiderstand ist.
Bei \SI{70}{\celsius} betr\"agt $\rho_T \approx 1.3$. $R_{PD}$ und $R_{PU}$ sind
die  Ausgangsimpedanzen  vom  LT3741  und  betragen  \SI{1.3}{\ohm}   respektive
\SI{2.4}{\ohm}.

Der  Low-Side MOSFET sollte einen m\"oglichst kleinen $R_{DS_{(on)}}$ haben  und
ein    Total-Gate-Charge    $Q_C    \leq    \SI{30}{\nano\coulomb}$    besitzen.

Ein  weiterer Verlust sind die Schaltverl\"uste der internen  MOSFET-Treiber  im
LT3741. Die Total Gate Charge $Q_C$ muss  w\"ahrend  jedem  Zyklus  geladen  und
wieder   entladen   werden.   Diese   Verl\"uste   k\"onnen   mit   der   Formel
\ref{eq:circuit:buck:switching_loss} berechnet werden,

\begin{equation}
    P_{LOSS\_LDO} \approx \left( (U_{in} - \SI{5}{\volt} \right) \cdot \left( Q_{GLG} + Q_{GHG} \right) \cdot f_S
    \label{eq:circuit:buck:switching_loss}
\end{equation}

wobei $G_{GLG}$ die  Low-Side  Gate-Charge $G_C$ ist und $G_{GHG}$ die High-Side
Gate-Charge ist.

In  der  Tabelle  \ref{tab:circuit:buck:mosfet}  sind  verschiedene MOSFET-typen
aufgelistet, die in den oben genannten Parametern passen. Dabei wurde $P_{LOSS}$
und $P_{LOSS\_LDO}$ f\"ur jeden Kandidaten berechnet.

\begin{table}[th!]
    \begin{center}
        \caption{}
        \label{tab:circuit:buck:mosfet}
        \begin{tabular}{cccccccccc}
            \toprule
            $R_{DS_{(on)}}$ & $Q_{GD}$ & $Q_{GS}$ & $R_G$ & $U_{GS_{THR}}$ & Ohmic Loss & Transision Loss & Total Loss & Drive Loss \\
            \midrule
            0.0032          & 4        & 2.5      & 0.4   & 2.5            & 0.104      & 1.0296          & 1.1336     & 0.806 \\
            0.0039          & 7        & 9        & 2.4   & 3.3            & 0.12675    & 4.8384          & 4.96515    & 1.984 \\
            0.0042          & 7        & 9        & 2.4   & 3.3            & 0.1365     & 4.8384          & 4.9749     & 1.984 \\
            0.008           & 2        & 4.5      & 3     & 2              & 0.26       & 2.2464          & 2.5064     & 0.558 \\
            0.0067          & 5.3      & 3.9      & 1.5   & 1              & 0.21775    & 2.18592         & 2.40367    & 0.7998 \\
            \rowcolor{lightgray}
            0.0093          & 2        & 4.9      & 1     & 2              & 0.30225    & 1.39104         & 1.69329    & 1.488 \\
            0.019           & 8        & 4        & 1.3   & 2              & 0.6175     & 2.6784          & 3.2959     & 1.798 \\
            0.0095          & 7.5      & 6        & 1     & 3              & 0.30875    & 2.7216          & 3.03035    & 1.736 \\
            \bottomrule
        \end{tabular}
    \end{center}
\end{table}

Vergleicht  man  \emph{Total  Loss}  und  \emph{Drive Loss}, w\"ahre der oberste
MOSFET  geeigneter. Aus Kostengr\"unden und  generell  schlechter  Dokumentation
wurde  aber  der   n\"achst  bessere  MOSFET  gew\"ahlt  --  hier  mit  hellgrau
hervorgehoben.

Es  wird  f\"ur  den  High-Side  MOSFET wie auch f\"ur den Low-Side  MOSFET  der
gleiche Typ verwendet.

