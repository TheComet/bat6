In contrast to a non-synchronous regulator, this design uses  two  complementary
MOSFETs $V_2$ and $V_3$, where $V_3$ acts as an active replacement for the  free
wheeling diode typically found in non-synchronous designs. As mentioned earlier,
a crucial feature of  this  device  is the ability to \emph{absorb} power. $V_3$
makes  this  possible  because  it  is  able to regulate current in the opposite
direction through the inductor $L_1$.

When  selecting  switching  MOSFETs,  the  following  parameters are critical in
determining the best devices for a given application: $Q_G$ (Total Gate Charge),
$R_{DS_{(on)}}$ (On-Resistance), $Q_{GD}$ (Gate to Drain Charge), $Q_{GS}$ (Gate
to Source  Charge),  $R_G$ (Gate Resistance), $U_{GS}$ (gate-to-source voltage),
$U_{DS}$  (drain-to-source-voltage),  $I_{D_{max}}$  (peak  drain  current)  and
$U_{GS_{THR}}$ (gate threshold voltage).

The maximum drain current is equal to the previously  calculated  peak  inductor
current $I_{L_{1_{peak}}}$ in equation \ref{eq:circuit:buck:inductor_peak}.
\begin{equation}
    I_{D_{max}} = I_{L_{1_{peak}}} = I_O + \left(\frac{U_{in}\cdot U_{out} - U_{out}^2}{2\cdot f_S \cdot L_1 \cdot U_{in}}\right) = \SI{5.2}{\ampere}
    \label{eq:circuit:buck:mosfet_id}
\end{equation}
where $U_{in}$ is the  input  voltage  \SI{28}{\volt},  $U_{out}$  is the output
voltage at peak power (which exists at $U_{out} = \SI{14}{\volt}$), $f_S$ is the
switching  frequency  \SI{1}{\mega\hertz},  $L_1$  is the value of the  selected
inductor (\SI{22}{\micro\henry})  and  $I_O$  is  the  maximum  output  current,
assumed to be $I_O = \SI{5}{\ampere}$.

The  maximum drain-to-source voltage $U_{DS}$ must be  greater  than  the  input
voltage $U_{in} = \SI{28}{\volt}$,  including  transients.  We  selected MOSFETs
with $U_{DS} = \SI{40}{\volt}$.

The  signals  driving  the  gates  of  the MOSFETs have  a  maximum  voltage  of
\SI{5}{\volt}  with   respect  to  the  source.  During  start-up  and  recovery
conditions, the gate drive signals  may  be  as  low as \SI{3}{\volt}. To ensure
that the LT3741 recovers properly, the  maximum gate threshold voltage should be
less than \SI{2}{\volt}. For a robust design, the maximum gate-to-source voltage
$U_{GS}$ should be greater than \SI{7}{\volt}.

Power losses in the MOSFETs are related to the on-resistance $R_{DS{(on)}}$; the
transition  losses  related  to  the  gate   resistance   $R_G$;   gate-to-drain
capacitance  $Q_{GD}$ and gate-to-source capacitance $Q_{GS}$. Power loss to the
on-resistance is an  Ohmic loss, $I^2 R_{DS_{(on)}}$. The power loss in the high
side     MOSFET     $V_2$      can      be     approximated     with     formula
\ref{eq:circuit:buck:mosfet_ploss}.

\begin{multline}
    P_{LOSS} = (\textrm{ohmic loss}) + (\textrm{transission loss}) \\
             \approx \left( I_O^2 \cdot R_{DS_{(on)}} \cdot \rho_T \right)
                    + \left( \frac{U_{in} \cdot I_O}{\SI{5}{\volt}} \cdot \left(Q_{GD} + Q_{GS} \right) \cdot \left( 2 \cdot R_G + R_{PU} + R_{PD} \right) \cdot f_S \right) \\
    \label{eq:circuit:buck:mosfet_ploss}
\end{multline}
where  $\rho_T$ is a temperature-dependant term of the  MOSFET's  on-resistance.
Using \SI{70}{\degree C} as the maximum operating temperature, $\rho_T$ is
roughly  equal  to  $1.3$.  $R_{PD}$  and $R_{PU}$ are the LT3741 high side gate
driver   output   empedance,  \SI{1.3}{\ohm}  and  \SI{2.3}{\ohm}  respectively.

Another power loss related to switching MOSFET selection  is  the  power lost to
driving the gates. The total gate  charge, $Q_G$, must be charged and discharged
each switching cycle. The power is lost to the  internal  LDO within the LT3741.
the power lost to the charging of the gates is:
\begin{equation}
    P_{LOSS\_LDO} \approx \left( (U_{in} - \SI{5}{\volt} \right) \cdot \left( Q_{GLG} + Q_{GHG} \right) \cdot f_S
    \label{eq:circuit:buck:switching_loss}
\end{equation}
where $G_{GLG}$ is the low side gate charge  and $Q_{GHG}$ is the high side gate
charge.

In the table \ref{tab:circuit:buck:mosfet} are various candidates  that meet the
above  constraints.  For   each   candidate  the  power  losses  $P_{LOSS}$  and
$P_{LOSS\_LDO}$ was calculated.

\begin{table}[th!]
    \begin{center}
        \caption{}
        \label{tab:circuit:buck:mosfet}
        \begin{tabular}{cccccccccc}
            \toprule
            $R_{DS_{(on)}}$ & $Q_{GD}$ & $Q_{GS}$ & $R_G$ & $U_{GS_{THR}}$ & Ohmic Loss & Transision Loss & Total Loss & Drive Loss \\
            \midrule
            0.0032          & 4        & 2.5      & 0.4   & 2.5            & 0.104      & 1.0296          & 1.1336     & 0.806 \\
            0.0039          & 7        & 9        & 2.4   & 3.3            & 0.12675    & 4.8384          & 4.96515    & 1.984 \\
            0.0042          & 7        & 9        & 2.4   & 3.3            & 0.1365     & 4.8384          & 4.9749     & 1.984 \\
            0.008           & 2        & 4.5      & 3     & 2              & 0.26       & 2.2464          & 2.5064     & 0.558 \\
            0.0067          & 5.3      & 3.9      & 1.5   & 1              & 0.21775    & 2.18592         & 2.40367    & 0.7998 \\
            \rowcolor{lightgray}
            0.0093          & 2        & 4.9      & 1     & 2              & 0.30225    & 1.39104         & 1.69329    & 1.488 \\
            0.019           & 8        & 4        & 1.3   & 2              & 0.6175     & 2.6784          & 3.2959     & 1.798 \\
            0.0095          & 7.5      & 6        & 1     & 3              & 0.30875    & 2.7216          & 3.03035    & 1.736 \\
            \bottomrule
        \end{tabular}
    \end{center}
\end{table}

The MOSFET highlighted in grey is the one we selected. In  this case, it is only
the second best candidate that fits the required parameters, but  it  is  a  lot
cheaper than the best fit and has better documentation.

The same device is used for both the high-side and low-side switch.

