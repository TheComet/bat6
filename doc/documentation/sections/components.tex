Es wird  in  Detail  beschrieben,  welche  elektronische Komponenten aus welchem
Grund ausgew\"ahlt worden sind.

% **************************************************************************** %
\subsection{36V Netzteil und Netzeingang}
% **************************************************************************** %

Die  maximale Ausgangsleistung wurde grob berechnet  mit  $\SI{24}{\volt}  \cdot
\SI{3}{\ampere} = \SI{72}{\watt}$.

Da  das  Aufbauen  eines  eigenen  Netzteils  f\"ur  diese  Ausgangsleistung  zu
aufw\"andig und teuer gewesen w\"are, entschieden  wir  uns  f\"ur  ein externes
Netzger\"at dass im Geh\"ause  montiert  werden  kann.  Das  verwendete Netzteil
liefert  \SI{36}{\volt}   und   \SI{75}{\watt}   und   ist   in   der  Abbildung
\ref{fig:circuit:mains-input} als $N_2$ zu sehen.

\begin{figure}[th!]
    \center
    \includegraphics[width=.35\textwidth]{images/circuit/mains-input.pdf}
    \caption{Netzspannung wird gefiltert und auf 36V DC durch ein externes Netzmodul transformiert}
    \label{fig:circuit:mains-input}
\end{figure}

Weiter  wird  eine  Netzeingangs-Steckverbinder  mit integriertem Netzfilter und
Sicherung  verwendet,  was  in der Abbildung  \ref{fig:circuit:mains-input}  als
$X_1$  zu  sehen  ist. Ein auf der R\"uckseit des Geh\"auses montierter Schalter
$X_7$ erlaubt das Ein- und Ausschalten des Endproduktes.

% **************************************************************************** %
\subsection{Spannungsversorgungen}
% **************************************************************************** %

F\"ur die digitale  Logik  werden  die  zwei  Spannungspegel  \SI{5}{\volt}  und
\SI{3.3}{\volt} ben\"otigt. 

\begin{figure}[th!]
    \center
    \includegraphics[width=.75\textwidth]{images/circuit/5v-3v-rails.pdf}
    \caption{Speisung f\"ur 5V mittels Abwertswandler (links) und Speisung f\"ur 3.3V mittels Linearregler (rechts)}
    \label{fig:circuit:rails}
\end{figure}

Die  \SI{36}{\volt} vom Netzteil werden mittels eines getakteten  DC-DC-Wandlers
auf \SI{5}{\volt} transformiert,  was  in  der Abbildung \ref{fig:circuit:rails}
vom Bauteil $N_3$ verwirklicht wird.

Die  \SI{5}{\volt}  werden  von  einem  Linearregler $N_4$  auf  \SI{3.3}{\volt}
gestuft.  Ein  Linearregler  wurde  gew\"ahlt damit die \SI{3.3}{\volt} Speisung
m\"oglichst St\"orfrei bleibt.  Somit  wird  verhindert,  dass die DACs und ADCs
verrauscht sind und ungenau messen.

% **************************************************************************** %
\subsection{LT3741}
% **************************************************************************** %

Die  Ausgangsspannung   muss   mindestens   im  Bereich  von  \SI{0}{\volt}  bis
\SI{24}{\volt}  liegen  und   einen  Rippel  kleiner  als  \SI{300}{\milli\volt}
besitzen.  Der  Ausgangsstrom muss mindestens im Bereich von \SI{0}{\ampere} bis
\SI{3.5}{\ampere} liegen  und  einen  Rippel kleiner als \SI{100}{\milli\ampere}
besitzen. Dabei  sollte  die Effizienz bei Volllast mindestens \SI{80}{\percent}
betragen.

Da  das  Endprodukt  schlussendlich  in  Serie  mit  mehreren   Spannungs-  oder
Stromquellen  geschalten  werden  k\"onnte, muss  zus\"atzlich  darauf  geachtet
werden,  dass  der Spannungswandler \emph{leistungsaufnahmef\"ahig}  sein  muss.
Diese Eigenschaft weist  ein  sogenannter  \emph{Synchronwandler}  vor und wurde
zusammen mit den Spannungs-,  Strom-,  und Leistungsanforderungen als prim\"ares
Merkmal f\"ur die Produktsuche eines Wandlers verwendet.

Der LT3741 erf\"ullt alle Anforderungen. In der Abbildung \ref{fig:circuit:buck}
ist der Aufbau zu sehen.

\begin{figure}[th!]
    \center
    \includegraphics[width=.75\textwidth]{images/circuit/buck.pdf}
    \caption{Herzst\"uck des Projektes: Aufbau des LT3741 CVCC Synchronwandler}
    \label{fig:circuit:buck}
\end{figure}

Der  LT3741  wird mit \SI{36}{\volt} gespiesen (oben  rechts  in  der  Abbildung
\ref{fig:circuit:buck}).   Da   diese   Schaltung   viel   St\"orung   auf   der
\SI{36}{\volt}    Speisung     verursachen    w\"urde,    werden    verschiedene
Keramikkondensatoren  und  Ferritkerne  verwendet,  um die hochfrequente Signale
m\"oglichst von der Speisung zu eliminieren und somit die anderen Speisungen von
ungewollten St\"orungen zu bewahren.

Die  Bauteile  $V_1$  und  $C_6$  bilden  zusammen  mit  dem  MOSFET  $V_2$  ein
High-Side-Schalter. Im Gegensatz zu einem nicht-synchronen Schaltregler befindet
sich  an  der Stelle wo normalerweise eine Freilaufdiode sein sollte ein zweiter
MOSFET $V_3$. Dieser erm\"oglicht  eine  Spannungsregelung in der gegengesetzten
Richtung -- sprich, sie erm\"oglicht eine Leistungsaufnahme, was, wie oben schon
erw\"ant wurde, kritisch ist.

Der   LT3741  ist  sowohl  Spannungsgesteuert  wie  auch   Stromgesteuert.   Der
Spannungsteiler  $R_{11}//R_2$  erlaubt das Messen der Ausgangsspannung und  ein
Shunt-Widerstand  $R_4$  erm\"oglicht die genaue \"Uberwachung des Stromes durch
die Spule $L_1$. Der Widerstand $R_4$ wurde  so  gew\"ahlt  damit  der  maximale
Ausgangsstrom \SI{5}{\ampere} betragen kann.

Strom\"uberwachung   ist  sehr  wichtig  bei  einer  solchen  Aufgabe   wo   die
Ausgangsspannung  sich  konstant  \"andert.  Sie erlaubt  genauer  vorhersebares
Verhalten der Spannungs\"anderung am Ausgang -- \"Uberschiessen der Sollspannung
und   extreme   Stromspitzen   in   der   Spule   k\"onnen   vermieden   werden.

Weiter   kann   ein   Stromgesteuerter   Regler  auch  als   Konstantstromquelle
funktionieren. Diese Eigenschaft  ist  vorallem  dann  von  Bedeutung  wenn  der
Arbeitspunkt  sich  im  ``steilen''  bereich  der   UI-Kennlinie  des  PV-Moduls
befindet.

Die   Widerst\"ande   $R_2$   und    $R_{11}$    wurden    nach    der    Formel
\ref{eq:circuit:buck:feedback_resistors}      dimensioniert      damit       die
Ausgangsspannung maximal \SI{23}{\volt} betr\"agt.

\begin{equation}
    U_{out} = \SI{1.21}{\volt} \left( 1 + \frac{R_{11}}{R_2} \right)
    \label{eq:circuit:buck:feedback_resistors}
\end{equation}

Die Ausgangsspannung kann  danach  durch Anheben der Bezugsspannung $BUCK\_USET$
nach der  Formel \ref{eq:circuit:buck:uset} ver\"andert werden. 

\begin{equation}
    U_{out} = (\SI{1.21}{\volt} - BUCK\_USET) \cdot \frac{R_{11} + R_2}{R_2}
    \label{eq:circuit:buck:uset}
\end{equation}

Wobei $BUCK\_USET$ die analoge Spannung vom DAC ist.

In der Abbildung \ref{fig:circuit:buck:uset}  ist  die dazugeh\"orige Schaltung.

\begin{figure}[th!]
    \center
    \includegraphics[width=.35\textwidth]{images/circuit/buck-uset.pdf}
    \caption{Einstellung der Ausgangsspannung durch \"Anderung der Bezugsspannung im Feedback-Loop mittels einer analogen Steuerspannung von 0V bis 1.21V}
    \label{fig:circuit:buck:uset}
\end{figure}

Durch anlegen einer analogen Spannung zwischen \SI{0}{\volt} und \SI{1.5}{\volt}
am Eingang  CTRL1  kann  direkt der maximale \emph{Durchschnittsstrom} durch die
Spule $L_1$ und  somit der maximale Ausgangsstrom eingestellt werden. Somit kann
bei  hoher Last der Wandler  auch  als  Konstantstromquelle  funktionieren.  Die
Abbildung  \ref{fig:circuit:buck:iset}  zeigt  die   dazugeh\"orige   Schaltung.

\begin{figure}[H]
    \center
    \includegraphics[width=.4\textwidth]{images/circuit/buck-iset.pdf}
    \caption{Einstellung des Maximalstroms mittels einer analogen Steuerspannung von 0V bis 1.5V}
    \label{fig:circuit:buck:iset}
\end{figure}

Dabei   kann  der  durchschnittliche  Ausgangsstrom   $I_o$   mit   der   Formel
\ref{eq:circuit:buck:output_current} berechnet werden

\begin{equation}
    I_o = \frac{U_{CTRL1}}{30 \cdot R_4}
    \label{eq:circuit:buck:output_current}
\end{equation}

wobei  $U_{CTRL1}$  die  analoge  Steuerspannung  vom  DAC  ist  und  $R_4$  der
Shunt-Widerstand ist, welcher  in  der Abbildung \ref{fig:circuit:buck} zu sehen
ist.

\begin{figure}[H]
    \center
    \includegraphics[width=.45\textwidth]{images/circuit/buck-umeas.pdf}
    \caption{Messen der Ausgangsspannung}
    \label{fig:circuit:buck:umeas}
\end{figure}

\begin{figure}[H]
    \center
    \includegraphics[width=.85\textwidth]{images/circuit/buck-imeas.pdf}
    \caption{Messen des Ausgangsstromes}
    \label{fig:circuit:buck:imeas}
\end{figure}

\begin{figure}[H]
    \center
    \includegraphics[width=.35\textwidth]{images/circuit/output-connectors.pdf}
    \caption{Verpolungsschutz am Ausgang}
    \label{fig:circuit:output}
\end{figure}

\begin{figure}[H]
    \center
    \includegraphics[width=.6\textwidth]{images/circuit/uvlo.pdf}
    \caption{Under-Voltage Lock-Out (UVLO) erm\"oglicht ein kontrolliertes Ein- und Ausschalten des Reglers}
    \label{fig:circuit:uvlo}
\end{figure}

\begin{figure}[H]
    \center
    \includegraphics[width=.4\textwidth]{images/circuit/vref.pdf}
    \caption{1.5V Referenzspannung um die ADCs m\"oglichst in Full-Range betreiben zu k\"onnen}
    \label{fig:circuit:vref}
\end{figure}

\begin{figure}[H]
    \center
    \includegraphics[width=.4\textwidth]{images/circuit/pushbutton.pdf}
    \caption{Drehdruckknopf}
    \label{fig:circuit:pushbutton}
\end{figure}

