Komponentenwahl, wieso wurde was ausgew\"ahlt?

\begin{figure}[H]
    \center
    \includegraphics[width=.35\textwidth]{images/circuit/mains-input.pdf}
    \caption{Netzspannung wird gefiltert und auf 36V DC transformiert}
    \label{fig:circuit:mains-input}
\end{figure}

\begin{figure}[H]
    \center
    \includegraphics[width=.75\textwidth]{images/circuit/5v-3v-rails.pdf}
    \caption{Speisung f\"ur 5V mittels Abwertswandler (links) und Speisung f\"ur 3.3V mittels Linearregler (rechts)}
    \label{fig:circuit:rails}
\end{figure}

\begin{figure}[H]
    \center
    \includegraphics[width=.75\textwidth]{images/circuit/buck.pdf}
    \caption{Herzst\"uck des Projektes: Aufbau des LT3741 CVCC Synchronwandler}
    \label{fig:circuit:buck}
\end{figure}

\begin{figure}[H]
    \center
    \includegraphics[width=.35\textwidth]{images/circuit/buck-uset.pdf}
    \caption{Einstellung der Ausgangsspannung durch \"Anderung der Bezugsspannung im Feedback-Loop mittels einer analogen Steuerspannung von 0V bis 1.5V}
    \label{fig:circuit:buck:uset}
\end{figure}

\begin{figure}[H]
    \center
    \includegraphics[width=.4\textwidth]{images/circuit/buck-iset.pdf}
    \caption{Einstellung des Maximalstroms mittels einer analogen Steuerspannung von 0V bis 1.5V}
    \label{fig:circuit:buck:iset}
\end{figure}

\begin{figure}[H]
    \center
    \includegraphics[width=.45\textwidth]{images/circuit/buck-umeas.pdf}
    \caption{Messen der Ausgangsspannung}
    \label{fig:circuit:buck:umeas}
\end{figure}

\begin{figure}[H]
    \center
    \includegraphics[width=.85\textwidth]{images/circuit/buck-imeas.pdf}
    \caption{Messen des Ausgangsstromes}
    \label{fig:circuit:buck:imeas}
\end{figure}

\begin{figure}[H]
    \center
    \includegraphics[width=.35\textwidth]{images/circuit/output-connectors.pdf}
    \caption{Verpolungsschutz am Ausgang}
    \label{fig:circuit:output}
\end{figure}

\begin{figure}[H]
    \center
    \includegraphics[width=.6\textwidth]{images/circuit/uvlo.pdf}
    \caption{Under-Voltage Lock-Out (UVLO) erm\"oglicht ein kontrolliertes Ein- und Ausschalten des Reglers}
    \label{fig:circuit:uvlo}
\end{figure}

\begin{figure}[H]
    \center
    \includegraphics[width=.4\textwidth]{images/circuit/vref.pdf}
    \caption{1.5V Referenzspannung um die ADCs m\"oglichst in Full-Range betreiben zu k\"onnen}
    \label{fig:circuit:vref}
\end{figure}

\begin{figure}[H]
    \center
    \includegraphics[width=.4\textwidth]{images/circuit/pushbutton.pdf}
    \caption{Drehdruckknopf}
    \label{fig:circuit:pushbutton}
\end{figure}

